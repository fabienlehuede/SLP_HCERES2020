\renewcommand{\nomequipe}{SLP}

\chapter{Équipe \nomequipe}

 \section{Présentation de l'équipe}
 
  	\subsection{Historique, localisation de l'équipe}
  
  		   %%%% ne pas supprimer les commentaires ; si vous désirez un pdf sans les commentaires mettre false à la place de true dans \setboolean{comment}{true} en debut de fichier rapportEquipe.tex à la racine du dossier HCERES   %%%%%
		   
  	
\CommentairesDirection {
  Rappel de l'historique IRCCyN et/ou LINA pour les équipes créées en 2017. 
  Cas particulier des équipes créées en cours de mandat : expliquer notamment la reprise de l'héritage en ce qui concerne les
  indicateurs. Présenter la localisation de l'équipe (et la répartition pour les équipes multi-sites). 
}

  
  	\subsection{Effectifs}
  	
   \CommentairesDirection{
  Le tableau des permanents sera automatiquement généré. Il contient les colonnes suivantes : Nom, prénom, statut, employeur, HdR, section CNU ou CoNRS ou ? \\
  pour IMT-A. Un renvoi sera possible pour indiquer du temps partiel ou du partage entre équipes. A la suite de ce tableau, dans un texte libre, l'équipe devra indiquer la taille
  de l'équipe en indiquant le nombre de non-permanents (en juin 2020) en distinguant doctorants, post-doctorants, ingénieurs, invités, collaborateurs... : ce devra être conforme aux données RH demandées. \\ 
  Les évolutions ou non des effectifs au cours de la  période devront être présentées. 
}
  
 
		  %%%% tableau généré par le fichier csv dans /equipes/equipe/donnees/personnelPermamentequipe.csv %%%%%
		   %%%% ne rien modifier %%%%%
		   
			  \DTLloaddb{persPerm\nomequipe}{./equipes/\nomequipe/donnees/personnelPermament\nomequipe.csv}
			%\restylefloat{table}
			\begin{table}[H]
			\begin{TableauSix}{Personnels permanents}
			\toprule
			\bfseries Nom &
			\bfseries Prénom &
			\bfseries Statut &
			\bfseries Employeur &
			\bfseries HDR &
			\bfseries Section CNU/CoNRS
			\DTLforeach*{persPerm\nomequipe}{\nom=Nom,\Prenom=Prenom,\Statut=Statut,\Employeur=Employeur,\HDR=HDR,\CNU=CNU}
			{
				\\\DTLifoddrow{\rowcolor{tcC}}{\rowcolor{tcB}}
				\nom & \Prenom & \Statut & \Employeur & \HDR & \CNU
			} % end of loop1
			\end{TableauSix}
			\end{table} 
		   
		   %%%% fin insertion automatique données tableau %%%%%
		   

  	\subsection{Politique scientifique}
  
   \CommentairesDirection{
  Ce paragraphe rassemble les éléments de politique et de positionnement au sein de l'écosystème de l'équipe (local, national ou international). On imagine le rappel des enjeux et défis abordés, le
  type de profil de l'équipe (incluant recherche amont, expérimentale,  valorisation). Les différents axes de l'équipe peuvent être  présentés. \\
  Ne pas oublier la prise en compte (ou non) des recommandations de la dernière évaluation. 
}

   
  \vfill
\rule[\baselineskip]{0pt}{\baselineskip}
 \section{Produits et activités de recherche}
 	
\subsection{Bilan scientifique}
  		
  FL: Ici réaliser un bilan par axe 
  
  Test citation: \cite{zhu:hal-01977572}.
  Remarque: pas d'année ???
  
\subsubsection{Maîtrise des Risques pour les Systèmes Industriels et les Services}

Dans cette axe, nous nous sommes tout d'abord intéressés a des
développements et à l'évaluation de méthodes de surveillance
(monitoring):
\begin{itemize}
\item pour des \emph{statistiques non usuelles} comme le coefficient
  de variation (\cite{yeong:hal-01716541},
  \cite{amdouni:hal-01573597}, \cite{you:hal-01351488},
  \cite{teoh:hal-01381673}, \cite{amdouni:hal-01388503},
  \cite{castagliola:hal-01083082}, \cite{castagliola:hal-01202423},
  \cite{amdouni:hal-01202424}) avec notamment les thèses de A. Achouri
  et A. Amdouni et son extension multivariée
  (\cite{khatun:hal-02072237}, \cite{nguyen:hal-01885435},
  \cite{ginerbosch:hal-02283481}, \cite{khaw:hal-01895172},
  \cite{yeong:hal-01307037}), le ratio de deux variables qualité
  normales (\cite{tran:hal-01773342}, \cite{celano:hal-01264243},
  \cite{celano:hal-01272650}, \cite{tran:hal-01308072},
  \cite{tran:hal-01345854}, \cite{tran:hal-01396009}) avec la thèse de
  K.P. Tran et son extension à $p$ variables via la notion de données
  compositionnelles (\cite{zaidi:hal-02273060},
  \cite{tran:hal-01731361}) avec la thèse de F. Zaidi.
\item pour des \emph{données de type fiabilité}
  (\cite{haghighi:hal-02183425}, \cite{castagliola:hal-01204508},
  \cite{haghighi:hal-01264248}) ou de type \emph{temps entre
    évènements et amplitudes} (\cite{rahali:hal-02135093},
  \cite{qu:hal-01809531}, \cite{qu:hal-01895010}) avec la thèse de
  D. Rahali.
\item pour des \emph{données de type discrète} comme Zero Inflated
  Poisson, Zero Inflated Binomial, Geometrically Inflated Poisson,
  avec ou sans autocorrelation (\cite{rakitzis:hal-01718089},
  \cite{rakitzis:hal-01466793}, \cite{rakitzis:hal-01659129},
  \cite{bersimis:hal-01676808}, \cite{rakitzis:hal-01231367},
  \cite{rakitzis:hal-01331381}, \cite{rakitzis:hal-01345886},
  \cite{rakitzis:hal-01345887}, \cite{rakitzis:hal-01166916}) avec les
  travaux du post-doctorant A. Rakitzis
\item pour des \emph{données auto-corrélées}
  (\cite{garzavenegas:hal-01824643}, \cite{franco:hal-01169579}) avec
  les travail du doctorant invité J.A. Garza Venegas,
\item pour des \emph{données de type profil}, dont la variable
  explicative dépend d'une ou plusieurs variables externes,
  (\cite{maleki:hal-02183432}, \cite{maleki:hal-01870773},
  \cite{maleki:hal-01905445}, \cite{guevara:hal-01580777}) avec
  notamment les travaux du doctorant invité R. Maleki.
\item pour des \emph{procédés à horizon de production fini}
  nécessitant la définition d'indicateurs de performance spécifiques
  (\cite{chong:hal-01978750}, \cite{celano:hal-01921379},
  \cite{celano:hal-01921386}, \cite{nenes:hal-01425549},
  \cite{celano:hal-01313577}, \cite{celano:hal-01367280},
  \cite{celano:hal-01382239}),
\end{itemize}

Nous nous sommes aussi intéressés à évaluer les performances de
méthodes de surveillance statistique de procédés dans le cas où:
\begin{itemize}
\item les paramètres nominaux (sous-contrôle) ont été \emph{estimés}
  et ne sont donc pas exactement connus (\cite{chong:hal-01978769},
  \cite{tang:hal-02015138}, \cite{tang:hal-02145998},
  \cite{hu:hal-02155302}, \cite{hu:hal-02160997},
  \cite{castagliola:hal-02190734}, \cite{hu:hal-02318364},
  \cite{khoo:hal-02354567}, \cite{hu:hal-01835841},
  \cite{oprime:hal-01421764}, \cite{you:hal-01614074},
  \cite{teoh:hal-01583912}, \cite{hu:hal-01668745},
  \cite{wu:hal-01286478}, \cite{teoh:hal-01326056},
  \cite{castagliola:hal-01349529}, \cite{yeong:hal-01180304},
  \cite{teoh:hal-01185340}, \cite{you:hal-01216234},
  \cite{you:hal-01348052}) avec les travaux des doctorants invités
  X. Hu et A.A. Tang,
\item une \emph{erreur de mesure} non négligeable doit être prise en
  compte (\cite{sabahno:hal-01977765}, \cite{tang:hal-01978754},
  \cite{sabahno:hal-02190729}, \cite{tang:hal-01806538},
  \cite{sabahno:hal-01921390}, \cite{maleki:hal-01423386},
  \cite{tran:hal-01668732}, \cite{hu:hal-01307056},
  \cite{hu:hal-01327313}, \cite{hu:hal-01396018},
  \cite{hu:hal-01228430}), avec la contribution des doctorants invités
  A. Maleki et H. Sabahno,
\item le calcul des propriétés associés au Run Length doivent être
  calculé de manière exacte, sans approximation ou méthodes de
  simulation (\cite{castagliola:hal-02002980},
  \cite{maravelakis:hal-02022526}, \cite{tang:hal-02059892},
  \cite{khoo:hal-01354062}). Il faut noter que
  \cite{castagliola:hal-02002980} est le second article publié dans la
  revue Journal of Quality Technology par l'auteur principal et qu'il
  est le seul auteur français à avoir réussi à publier deux articles
  dans cette revue très sélective.
\end{itemize}

Finalement, nous avons contribué à résoudre des problématiques de
dégradation et de maintenance en grandes dimensions appliquées, en
particulier, à des infrastructures du génie civil et utilisant, entre
autre, des méthodes de type Réseaux Bayésiens (RB). Nous nous sommes
plus particulièrement intéressés à:
\begin{itemize}
\item la modélisation pure du phénomène de détérioration et de
  l'aspect cyclique de la phase de maintenance
  (\cite{kosgodagan:hal-01517154}) et la proposition d'un modèle de RB
  dynamique afin de fournir des stratégies de coûts optimisés dans le
  contexte de la gestion de réseaux de ponts routiers
  (\cite{kosgodagandallatorre:hal-01517168}).
\item une extension du RB dynamique standard en introduisant une
  dimension supplémentaire qui prend en compte des ``éléments''
  spatiaux qui sont ainsi liés par un ensemble de covariables qui
  traduisent les dépendances probabilistes existantes
  (\cite{kosgodagan:hal-01517174}). Ce point et le précédent
  correspondent aux travaux de thèse d'A. Kosgodagan.
\item une approche basée sur les RB discrets pour quantifier les
  probabilités de transition de l'état du système
  (\cite{acharige:hal-01152564}),
\item la proposition d'un premier modèle qui considère l'option ou la
  possibilité d'acquérir une information sur la rentabilité de
  nouvelle technologie pas encore disponible sur le marché pour la
  prise de décision de maintenance et remplacement des biens
  (\cite{nguyen:hal-01520843}) avec la thèse de T.P.K. Nguyen,
\item une méthodologie pour la formulation et la construction d'un
  méta-modèle qui intègre des données issues des mesures de contrôle
  non destructif et de l'évolution de la pathologie de dégradation
  choisie (\cite{elhajj:hal-01520822}, \cite{elhajj:hal-01316236})
  avec la thèse de B. El Hajj.
\end{itemize}

  \subsubsection{Axe Conception, planification et ordonnancement des systèmes de production et de services}
  
La recherche menée dans cet axe de l’équipe SLP se structure autour de cinq thématiques majeures : l’ordonnancement, la planification de personnel, la planification de production, la conception de ligne d’assemblage et l’équilibrage de la charge des lignes de production. \\

Concernant l’ordonnancement, les travaux menés se sont focalisés sur la détermination de la complexité de problèmes d’ordonnancement théorique. Ces travaux ont été menés lors de la thèse de doctorat de Tianyu Wang grâce à un financement « Chinese Scholar Council » \cite{wang:hal-02062182,wang:hal-02062178}.  La recherche en planification de personnel a pris comme cadre d’application les systèmes hospitaliers : un projet cofinancé par Atlanstic, l’université d’Angers et IMT Atlantique a permis le démarrage d’une thèse de Doctorat sur le sujet. Il est à noter le dépôt d’un ouvrage aux presses des Mines autour de la planification de personnel. Des travaux autour de la planification des blocs opératoires ont également été menés lors de la thèse de doctorat de Hasan Al Hasan en collaboration avec Université d’Angers et l’Université Catholique de l’Ouest \cite{alhasan:hal-01875359}. \\

Pour la planification de production, les travaux se sont majoritairement concentrés sur deux directions : la planification sous incertitude et la planification financière. La première consiste à définir les quantités à produire ou à commander afin de pouvoir satisfaire la demande en produit fini pour minimiser les coûts logistiques alors que des paramètres tels que la demande, la capacité et le lead time sont incertains. Des modèles mathématiques stochastiques et des méthodes de résolutions efficaces ont été développées pour y répondre \cite{benammar:hal-02415341,benammar:hal-02435962,benammar:hal-01961194,BENAMMAR201839,Thevenin2020,borodin:hal-01313213}. La thématique de la planification financière a été abordée lors du projet FUI RCSM et trouve sa continuation dans le cadre du projet ANR FILEAS FOG mené conjointement avec l’IGR-IAE de Rennes. Le but de cette recherche est d'étudier l’impact des décisions financières sur les décisions de planification de la production, et, plus généralement, les décisions logistiques.  Cette recherche a donné lieu à une première thèse de doctorat soutenue par Yuan Bian \cite{bian:hal-02190123,bian:halshs-01683781} et une seconde a débuté en 2018.  Des travaux sur l’intégration de décisions sont également menés : on peut citer par exemple l’intégration de la planification de la production avec les décisions d’inspection \cite{bettayeb:hal-01689377} et la problématique d’intégration des décisions de planification et d’ordonnancement qui ont donné lieu au démarrage d’une thèse CIFRE conjointement avec la société VIF. Outre cela, on peut citer également le développement d’outils de conception/planification et ordonnancement sous incertitudes basées sur des méthodes de programmation stochastique ou robuste, ou encore la résolution du problème de planification intégrée de stock de produit vrac, de routage et d’ordonnancement d’accostage des navires dans un port de vrac dans le cadre d'une collaboration avec l’Université UM6P de Benguerir au Maroc \cite{Najid2020}. \\

La recherche sur la conception combinatoire des systèmes de production reconfigurables est un autre sujet de recherche développé dans l’axe. En se basant sur des résultats obtenus précédemment pour la conception optimale des lignes de transfert pour la fabrication de grandes séries, des modèles et techniques d'optimisation permettant de choisir des équipements et de configurer optimalement des lignes de production reconfigurables pouvant fabriquer plusieurs types de produits ont été développés \cite{Battaia2020,battaia:hal-01435089,battaia:hal-01523726,Dolgui2019,Dolgui2020, yelleschaouche:hal-02485673}. \\

Enfin, la dernière thématique concerne l’équilibrage des lignes de production. Celle-ci a été étudiée dans un cas déterministe, en tenant compte des variations de charge induites par les changements de demande, d’un portefeuille multi-produit etc. Des études portant sur la complexité algorithmique de variantes de cette problématique ont été menées et des algorithmes efficaces ont été développés \cite{delorme:emse-01840007,dolgui:hal-01688688}. Cette problématique a également été abordée sous l’angle de la robustesse en utilisant un indicateur appelé « rayon de stabilité » que l’on cherche à maximiser. Les résultats obtenus portent essentiellement sur le développement de méthodes de résolution exactes efficaces à l’aide de la programmation linéaire en nombres mixtes (PLNM) \cite{rossi:hal-01301625}, basée sur la génération de coupes optimums, la réduction de l’espace de recherche grâce à des inégalités valides \cite{pirogov:hal-01832920,pirogov:hal-01713722, pirogov:hal-01614455} ou bien sur la décomposition de Dantzig-Wolf pour de nouvelles variantes du problème de bin-packing qui en découlent \cite{schepler:hal-01518356,schepler:hal-02485676, schepler:hal-01474542}. Une thèse de doctorat a été également soutenu en 2019 sur ce sujet \cite{pirogov:tel-02418792}.
  

  \subsubsection{Axe Conception et optimisation des réseaux logistiques et de transport}
  		
   
Le troisième axe de l'équipe SLP porte sur le développement de modèles et algorithmes d'optimisation pour la conception de réseaux logistiques et la résolution de problèmes de trounées de véhicules (VRP: Vehicle Routing Problem).

Sur le plan scientifique, l'équipe a obtenu des résultats sur les problématiques suivantes : 
\begin{itemize}
	\item Sur les \textit{tournées de véhicules avec synchronisation}, l'équipe a réalisé de fortes avancées sur le développement de la méta-heuristique Large Neighborhood Search (LNS), de procédures de réalisabilité efficaces et sur l'hybridation de LNS avec des approches exactes. Ainsi, en collaboration avec le CIRRELT de Montréal, l'équipe a proposé le premier algorithme intégrant des fenêtre de temps dans le problème de tournées de véhicules (VRP) à deux échelons \cite{grangier2015}. La notion de capacité des points de transfert a été introduite dans le VRP avec cross-docking \cite{gangier206,grangier2019} et dans la planification de transports en camion pleins pour les cas de capacité un \cite{grimaultCOR,grimaultThese}. Ces algorithmes ont été appliqué au transport mixte et multi-modal \cite{masson2017}. 
	ici : thèses HDR projets ?
	\item Pour la conception de réseaux logistiques, l'équipe a réalisé des avancées sur deux principaux aspects: Tout d'abord, l'intégration de critères de développement durables dans la conception de la supply chain, avec un état de l'art \cite{eskandarpour2015}, ...
    En parallèle, l'équipe développe son expertise sur la conception de réseaux de transports mutualisés adaptés au secteur de l'horticulture \cite{xintang} avec le FUI Vegesupply, ou de l'approvisionnement des plate-formes de la grande distribution \cite{medina:hal-01689718,...}, à travers la thèse de Juliette Medina, le PIA ADEME OPEN Network, la thèse de Gauthier Soleilhac (préciser les aspects scientifiques). 
    \item Plus généralement, l'équipe contribue à la résolution de problématiques de transport pour l'industrie et les service: sur le transport de personnes à mobilité réduite, de nouveaux algorithmes ont été proposés pour l'optimisation de tournées avec véhicules reconfigurables \cite{tellez:2019} et la construction de plans de transports réguliers \cite{tellezcondarp}.
    La collecte de déchets a été abordée dans la thèse CIFRE de Quentin Tonneau en collaboration avec l'entreprise Brangeon. 
    L'intégration de priorités est étudiée dans le cadre de la thèse de Tan Doan en collaboration avec l'université de Hannoï. 
\end{itemize}


Activité forte Projets \& animation scientifique

Tout en Nb de publis de rang A – + Omega  EJOR TRC COR

X thèses soutenues 1 HDR
Projets :  

Restant : Végésupply + These 4S CIFRE Brangeon, CIFRE Lumiplan, PHC PROCOPE ATITUDS , PIA ADEME CRC ON, FEDER NOMAD, ANR FILEAS FOG (inter-axe) ; ANR OPUSS ; CARENE OPT-EMR ; Post-Doc UBL; DISC (inter-axe ?); COLOUR; CIFRE CRC Services; CIFRE Lumiplan 2 ; TSL Cross-regional grant ; Mission exploration japon 2019

Collaborations internationales : CIRRELT, Mike Hewitt, Mayence, Linz, Mobilité Fabien, Jiao Tong Un., Hanoï, 

Nationale : Lyon, Rennes, Angers, 

Animation scientifique : VEROLOG 2016, NOW 2015, Synchrotrans 2019, ROADEF 2018, Board EWGLA 2015-2019, GT2L prix de thèse 2019, prix de thèse Verolog 2020. 		
  		
  		
  		
  \subsubsection{Axe Thèmes transversaux et fondamentaux}
  
L'axe 4 regroupe les travaux de l'équipe avec une coloration plutôt académique portant essentiellement, d'une part, sur l'étude de complexité algorithmique pour des problèmes classiques (ou leurs variantes) en optimisation combinatoire et, d'autre part, sur le développement de méthodes exactes en optimisation multi-objectif.      

\begin{enumerate}
    \item L’équipe a conduit une étude de complexité algorithmique pour des problèmes d'optimisation combinatoire avec données contrôlables et fonctions de risque linéaires (min-sum) et non-linéaires (min-max). Ses problèmes d’optimisation retrouvent des applications en conception du réseau et routage. Pour tous les problèmes traités, nous avons développé des algorithmes polynomiaux et pseudo-polynomiaux plus efficaces \cite{gurevsky:hal-01262639,gurevsky:hal-01272518,gurevsky:hal-01341259,gurevsky:hal-02504487}, en termes de l'ordre de grandeur, par rapport à ceux proposés auparavant dans la littérature. 
    
    \item L’activité sur le thème « optimisation multi-objectif » sur la période 2015-2019 a été naturellement rythmée par le projet ANR-DFG vOpt (Exact Efficient Solution of Mixed Integer Programming Problems with Multiple Objective Functions).


La principale méthodologie considérée dans ce projet pour la résolution exacte de problèmes d’optimisation multi-objectif a été le branch and bound multi-objectif. Un état de l’art détaillé au sujet de ce type de méthode \cite{przybylski:hal-01717951} et ensuite, des travaux raffinant chacun de ses composants ont été réalisés : ensembles bornants, choix de la variable de branchement, choix du noeud actif. 

%Une étude du lien entre l’enveloppe convexe d’un ensemble de points et la décomposition de l’ensemble des poids a permis d’identifier des conditions pour lesquelles des facettes définies par des points non-dominés supportés sont dominées. Ensuite, une méthode s’appuyant sur le calcul incrémental d’un ensemble de points non-dominés supportés, et se reposant fortement sur une initialisation appropriée dans le but d’éviter la considération de facettes dominées, a été proposée [2]. Une nouvelle méthode pour la résolution de la relaxation convexe passant par une décomposition de l’ensemble des poids a également été proposée [3].

Une première méthode pour calculer l’ensemble bornant optimal convexe surrogate a été proposée dans \cite{cerqueus:hal-01158355} et repose sur une décomposition de l’ensemble des multiplicateurs. Une nouvelle méthode \cite{przybylski:hal-02480174} a été proposée en réalisant une méthode d’analyse de sensibilité pour le problème dual-surrogate mono-objectif. Cet ensemble bornant supérieur est un outil puissant pour pré-traiter le problème du sac à dos bi-objectif bi-dimensionnel.

Le choix de la variable de branchement a été étudié pour le cas spécifique du sac à dos bi-objectif en variables binaires \cite{cerqueus:hal-01564982}. Le choix du noeud actif a été considéré avec une stratégie originale pour le problème de localisation de services bi-objectif \cite{delmee:hal-01435524} (avec et sans contrainte de capacité). Une autre méthode a  été également proposée pour la résolution exacte du problème avec contrainte de capacité : une méthode en deux phases pour laquelle la seconde phase se repose sur un algorithme de branch and bound \cite{delmee:hal-02480176}. 
\end{enumerate}
	







  		
\subsection{Faits marquants}



\begin{itemize}
    \item Aspects RH -- investissement des tutelles --  Recrutements : A. Dolgui, G. Massonnet, S. Thévenin
    \item Nb Publis de rang A
    \item Organisation VEROLOG 2016
    \item 2 projets ANR Commencés en 2018
    \item A. Dolgui -- IISE Fellow Award.
\end{itemize}


\section{Vie de l'équipe}

	

Resp. resp adjoint
organisation en séminaires (invités et membres externes)
Ressources sont affectées de manières collégiales
Ressource d’un projet à la responsabilité du resp de projet
Réunion d’équipe une fois par mois 

Parité ; Intégrité scientifique ; Hygiène et sécurité ; Développement durable et prise en compte des  
impacts environnementaux ;  Propriété intellectuelle et intelligence économique
