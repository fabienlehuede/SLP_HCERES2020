\renewcommand{\nomequipe}{SLP}

\chapter{Équipe \nomequipe}

 \section{Présentation de l'équipe}
 
 L'équipe SLP conçoit des modèles et algorithmes d'optimisation et aide à la décision pour la résolution de problèmes complexes. 
 Elle inscrit sa recherche dans les domaines de la recherche opérationnelle et de la décision dans l'incertain. 
 

 
  	\subsection{Historique, localisation de l'équipe}
  
  		   %%%% ne pas supprimer les commentaires ; si vous désirez un pdf sans les commentaires mettre false à la place de true dans \setboolean{comment}{true} en debut de fichier rapportEquipe.tex à la racine du dossier HCERES   %%%%%
		   
  	
\CommentairesDirection {
  Rappel de l'historique IRCCyN et/ou LINA pour les équipes créées en 2017. 
  Cas particulier des équipes créées en cours de mandat : expliquer notamment la reprise de l'héritage en ce qui concerne les
  indicateurs. Présenter la localisation de l'équipe (et la répartition pour les équipes multi-sites). 
}
L'équipe SLP a été fondée à la création de l'IRCCyN en 2000.
Elle comporte 15 permanents, dont 8 à IMT Atlantique et 7 à l'Université de Nantes, répartis sur les sites de la FST à Nantes et des IUT à Carquefou et Saint-Nazaire.
Elle accueille également un collaborateur extérieur, professeur associé à Rennes School of Business. 

L'équipe est dirigée par Fabien Lehuédé (responsable d'équipe) et Evgeny Gurevsky (adjoint). 



  	\subsection{Effectifs}
  	
   \CommentairesDirection{
  Le tableau des permanents sera automatiquement généré. Il contient les colonnes suivantes : Nom, prénom, statut, employeur, HdR, section CNU ou CoNRS ou ? \\
  pour IMT-A. Un renvoi sera possible pour indiquer du temps partiel ou du partage entre équipes. A la suite de ce tableau, dans un texte libre, l'équipe devra indiquer la taille
  de l'équipe en indiquant le nombre de non-permanents (en juin 2020) en distinguant doctorants, post-doctorants, ingénieurs, invités, collaborateurs... : ce devra être conforme aux données RH demandées. \\ 
  Les évolutions ou non des effectifs au cours de la  période devront être présentées. 
}
  
  L'équipe est composée de 15 permanents, un collaborateur extérieur, deux enseignants-checheurs en CDD (MAA et ATER), un post-doctorant et 17 doctorants.
  
  
		  %%%% tableau généré par le fichier csv dans /equipes/equipe/donnees/personnelPermamentequipe.csv %%%%%
		   %%%% ne rien modifier %%%%%
		   
			  \DTLloaddb{persPerm\nomequipe}{./equipes/\nomequipe/donnees/personnelPermament\nomequipe.csv}
			%\restylefloat{table}
			\begin{table}[H]
			\begin{TableauSix}{Personnels permanents}
			\toprule
			\bfseries Nom &
			\bfseries Prénom &
			\bfseries Statut &
			\bfseries Employeur &
			\bfseries HDR &
			\bfseries Section CNU/CoNRS
			\DTLforeach*{persPerm\nomequipe}{\nom=Nom,\Prenom=Prenom,\Statut=Statut,\Employeur=Employeur,\HDR=HDR,\CNU=CNU}
			{
				\\\DTLifoddrow{\rowcolor{tcC}}{\rowcolor{tcB}}
				\nom & \Prenom & \Statut & \Employeur & \HDR & \CNU
			} % end of loop1
			\end{TableauSix}
			\end{table} 
		   
		   %%%% fin insertion automatique données tableau %%%%%
		   

Sur la période 2015--2020, SLP a connu d'importants mouvements de personnels, principalement à IMT-Atlantique. Alexandre Dolgui est arrivé au poste de Professeur et chef du Département Automatique Productique Informatique en 2015. Guillaume Massonnet et Simon Thévenin ont été embauchés comme Maîtres-Assistants respectivement en 2016 et 2018. 
Le Professeur Pierre Dejax a pris sa retraite en 2018. 
Deux Maîtres-Assistants de l'équipe ont été promus au poste de Professeur: Olivier Péton en 2017 et Fabien Lehuédé en 2019. 
Une embauche est planifiée en 2020 pour l'équipe SLP ou l'équipe TASC sur le profil ``Optimisation et IA pour le Transport et la Logistique''. 

Les arrivées compensent principalement des départs ayant eu lieu au précédent quadriennal. Ces postes sont néanmoins révélateurs de l'intérêt d'IMT Atlantique et du LS2N pour la thématique de l'Industrie du Futur et du Génie Industriel. 
De plus, l'arrivée d'Alexandre Dolgui dans l'équipe est bénéfique pour sa visibilité.
Le dernier poste est motivé par l'ouverture en Septembre 2020 d'une formation d'ingénieur en alternance sur la transition numérique. \textcolor{olive}{OP: donner intitulé exact} Cette formation en 3 ans devrait générer de nouveaux besoins en enseignants--chercheurs dans le domaine de l'équipe SLP entre autres. 

Deux HDR ont été soutenues en 2015, une soutenance supplémentaire est planifiée pour 2020.

Entre 2015 et 2020,  six enseignants-chercheurs ont été recrutés sur des CDD de Maître-Assistant Associé de un à trois ans à IMT Atlantique. 
L'Université de Nantes a recruté deux ATER sur des postes de un et deux ans.
Sept post-doctorats ont été réalisés.

19 thèses encadrées par l'équipe ont été soutenues, avec une moyenne de 40 mois par thèse. 
%Quatre doctorants ont abandonné leur thèse. \textcolor{olive}{C'est demandé ? T'es obligé de le dire ?} 

% \textit{FL : est-ce qu'on met quelques mots pour expliquer les abandons ? Dépression c'est pas très gai...}
%Ka Yu Lee a terminé son contrat CIFRE et n'a pas souhaité finaliser son rapport de fin de contrat sous forme de mémoire afin de partir vers le privé. 
% Olivier Bachollet, en co-tutelle avec Montréal, a abandonné pour des problèmes de santé. 
%...

Sur les thèses démarrées après 2015, on compte 11 thèses financées sur projet ou en CIFRE, 8 thèses financées ou co-financées par une bourse MNRT  ou une tutelle,
6 thèses financées par des gouvernements étrangers.
Les co-financements et co-encadrements se font majoritairement avec l'université d'Angers, le CIRRELT ou IVADO à Montréal et Rennes School of Business. 

%\textit{ éventuellement parler des doctorants inscrits ailleurs et co-encadrés ici ?
%}

\subsection{Politique scientifique}
  
   \CommentairesDirection{
  Ce paragraphe rassemble les éléments de politique et de positionnement au sein de l'écosystème de l'équipe (local, national ou international). On imagine le rappel des enjeux et défis abordés, le
  type de profil de l'équipe (incluant recherche amont, expérimentale,  valorisation). Les différents axes de l'équipe peuvent être  présentés. \\
  Ne pas oublier la prise en compte (ou non) des recommandations de la dernière évaluation. 
}

L'équipe SLP développe des méthodes analytiques pour l'optimisation et l'aide à la décision en production et logistique. 
Elle réalise une recherche en recherche opérationnelle et décision dans l'incertain qui trouve ses applications au coeur de nombreux défis sociétaux.
Nos résultats sont en particulier appliqués à l'industrie du futur ; la planification des services et en particuliers des services de soin ou d'assistance aux personnes ; la mobilité des personnes et des biens.
Les contributions sont de deux types : 
\begin{itemize}
\item \textbf{La modélisation ou résolution de nouveaux problèmes}. En premier lieux, ceux-ci sont  inspirés par l'apparition de nouveaux enjeux, de nouveaux équipements ou données, ou de nouvelles pratiques. 
Ils peuvent également être issus de l'intégration de fonctions traitées séparément afin d'établir une optimisation plus globale. 
\item \textbf{La résolution de problèmes académiques}. Celle-ci a pour objectif de résoudre des problèmes connus et encore ouverts, ou de proposer des modèles ou algorithmes plus adaptés ou efficaces.
\end{itemize}

Les travaux de l'équipe se présentent selon quatre axes:\\
%\begin{itemize}
%    \item 
\textbf{Axe maîtrise des risques pour les systèmes industriels et les services.} 
    Cet axe réalise une recherche en décision dans l'incertain basée sur les statistiques et les probabilités, combinés à des méthodes d'optimisation stochastique.
    Les domaines d'application visés sont principalement 
    la surveillance des procédés de production et de service ; et
    la modélisation et l'optimisation des politiques de maintenance.\\
%    \item 
\textbf{Axe conception, planification et ordonnancement des systèmes de production et de services.}
Cet axe s'intéresse aux nombreuses problématiques d'optimisation présentes dans les systèmes de production et les services. 
Les recherches portent notamment sur la conception de ligne de production, l'ordonnancement d'atelier et de projet, la planification de production et la planification de ressources humaines. 
Les applications apparaissent aussi bien dans l'industrie que dans les services (en particulier dans les systèmes de soins).\\
%    \item 
\textbf{Axe conception et optimisation des réseaux logistiques et de transport.}
Cet axe regroupe des recherches sur des problèmes de conception de réseaux (chaînes logistiques, réseaux de distribution) et sur leur optimisation opérationnelle (problèmes de tournées de véhicules). 
Les applications développées portent aussi bien sur le transport de matériaux et de marchandises qu'à la mobilité des personnes. \\
%    \item 
\textbf{Axe travaux fondamentaux.}
Cet axe regroupe les contributions théoriques indépendantes ou parallèles aux travaux menés dans les trois axes applicatifs de l'équipe. 
Il comporte notamment des contributions en optimisation multiobjectif et complexité des algorithmes. %\end{itemize}

L'ensemble des travaux réalisés sont valorisés à travers un important volume de publications en revues internationales de rang A.

\textbf{Recommandations de la précédente évaluation:}

En accord avec les précédentes recommandations, une plus grande visibilité de la vie de l'équipe est donnée en Section \ref{slp:vieequipe}.

Le rayonnement de l'équipe a été considérablement augmenté avec le recrutement d'Alexandre Dolgui, mais également avec l'organisation d'évènements majeurs de la communauté tels que l'organisation de la conférence VeRoLog 2016 ou la participation à l'organisation de la conférence Roadef 2018 à Lorient. 
L'équipe a également animé le GT EWGLA de l'association EURO ou le GT BERMUDES des GDR MACS et RO.
Pour les années à venir, l'équipe organise les conférences PMS 2021 et MIM 2022.
\textcolor{olive}{Ajouter soit un lien vers les GT et confs, soit l'intitulé complet}
\textcolor{blue}{Les liens complets c'est plutôt dans les annexes non ?}
Par ailleurs, l'obtention de deux ANR, la promotion au rang de Professeur de deux membres et l'ouverture de trois postes de Maîtres-Assistants est révélateur de la bonne dynamique interne de l'équipe.

%\textit{Interaction avec les autres équipes ... à développer.... est-ce complet ?}
Au niveau des collaborations internes au LS2N, elles ont principalement lieu en collaboration avec l'équipe PSI, sur les systèmes reconfigurables  \cite{lameche:hal-02354553, kouiss:hal-02354412}, ou sur l'évaluation de performance en simulation \cite{indriago:hal-01693153,indriago:hal-01628882}. 
La proximité d'une partie de l'équipe avec PSI a amené à proposer la reconfiguration présentée dans le projet.
Le projet FUI Hypperwind a été réalisé en partenariat avec l'équipe DUKE. Une réponse à l'appel Région PME a été déposé en collaboration avec cette équipe en 2020 \textcolor{blue}{revoir la formulation }. 
Le workshop aLife 
% sur l'apport de l’industrie du Logiciel à l’Industrie du Futur Européenne (\url{https://conferences.imt-atlantique.fr/alife/})
a été organisé en 2018 en collaboration avec les équipes NaoMod, STACK et TASC. 
Ce workshop a donné lieu à la rédaction d'un livre blanc intitulé Logiciel et Industrie du Futur \cite{bach:hal-02299214}. Des projets sur ce thèmes sont déposés en collaboration avec l'équipe TASC.

Concernant les spécificités de SLP, l'équipe a principalement travaillé au renforcement de son positionnement sur l'industrie du futur et l'optimisation dans un contexte dynamique et incertain. Dans ce cadre, elle a recruté deux MA avec un profil <<optimisation stochastique et/ou robuste en production>> et vient d'ouvrir un poste <<optimisation et IA pour le transport et la logistique>>. 
Au niveau interne, un groupe de lecture sur l'optimisation stochastique a été organisé et des projets ont été déposés (par exemple, le projet  DISC).
Cependant, l'équipe comporte des spécialistes internationalement reconnus dans des domaines variés mais ne possède peut être pas encore une identité propre. Une des raisons est probablement le temps nécessaire à la construction d'une dynamique de groupe avec les nouveaux arrivants, mais également la large taille et couverture de SLP. 
Nous pensons que la nouvelle configuration et le nouveau positionnement proposé dans le projet contribuera à mieux faire apparaître l'équipe dans le paysage scientifique. 

%Questions : 
%\begin{itemize}
%    \item devenir des doctorants ?
%    \item implication dans la formation par la recherche (master ORO, MOST, formations Ingé ?)
%\end{itemize}


  \vfill
\rule[\baselineskip]{0pt}{\baselineskip}
 \section{Produits et activités de recherche}
 	
\subsection{Bilan scientifique}

Le bilan scientifique est présenté selon les quatre axes de recherche de l'équipe. 		

\subsubsection{Maîtrise des risques pour les systèmes industriels et les services}

Dans cette axe, nous nous sommes tout d'abord intéressés a des
développements et à l'évaluation de méthodes de surveillance
(monitoring):
\begin{itemize}
\item pour des \emph{statistiques non usuelles} comme le coefficient
  de variation (\cite{yeong:hal-01716541},
  \cite{amdouni:hal-01573597}, \cite{you:hal-01351488},
  \cite{teoh:hal-01381673}, \cite{amdouni:hal-01388503},
  \cite{castagliola:hal-01083082}, \cite{castagliola:hal-01202423},
  \cite{amdouni:hal-01202424}) avec notamment les thèses de A. Achouri
  et A. Amdouni et son extension multivariée
  (\cite{khatun:hal-02072237}, \cite{nguyen:hal-01885435},
  \cite{ginerbosch:hal-02283481}, \cite{khaw:hal-01895172},
  \cite{yeong:hal-01307037}), le ratio de deux variables qualité
  normales (\cite{tran:hal-01773342}, \cite{celano:hal-01264243},
  \cite{celano:hal-01272650}, \cite{tran:hal-01308072},
  \cite{tran:hal-01345854}, \cite{tran:hal-01396009}) avec la thèse de
  K.P. Tran et son extension à $p$ variables via la notion de données
  compositionnelles (\cite{zaidi:hal-02273060},
  \cite{tran:hal-01731361}) avec la thèse de F. Zaidi.
\item pour des \emph{données de type fiabilité}
  (\cite{haghighi:hal-02183425}, \cite{castagliola:hal-01204508},
  \cite{haghighi:hal-01264248}) ou de type \emph{temps entre
    évènements et amplitudes} (\cite{rahali:hal-02135093},
  \cite{qu:hal-01809531}, \cite{qu:hal-01895010}) avec la thèse de
  D. Rahali.
\item pour des \emph{données de type discrète} comme Zero Inflated
  Poisson, Zero Inflated Binomial, Geometrically Inflated Poisson,
  avec ou sans autocorrelation (\cite{rakitzis:hal-01718089},
  \cite{rakitzis:hal-01466793}, \cite{rakitzis:hal-01659129},
  \cite{bersimis:hal-01676808}, \cite{rakitzis:hal-01231367},
  \cite{rakitzis:hal-01331381}, \cite{rakitzis:hal-01345886},
  \cite{rakitzis:hal-01345887}, \cite{rakitzis:hal-01166916}) avec les
  travaux du post-doctorant A. Rakitzis
\item pour des \emph{données auto-corrélées}
  (\cite{garzavenegas:hal-01824643}, \cite{franco:hal-01169579}) avec
  les travaux du doctorant invité J.A. Garza Venegas,
\item pour des \emph{données de type profil}, dont la variable
  explicative dépend d'une ou plusieurs variables externes,
  (\cite{maleki:hal-02183432}, \cite{maleki:hal-01870773},
  \cite{maleki:hal-01905445}, \cite{guevara:hal-01580777}) avec
  notamment les travaux du doctorant invité R. Maleki.
\item pour des \emph{procédés à horizon de production fini}
  nécessitant la définition d'indicateurs de performance spécifiques
  (\cite{chong:hal-01978750}, \cite{celano:hal-01921379},
  \cite{celano:hal-01921386}, \cite{nenes:hal-01425549},
  \cite{celano:hal-01313577}, \cite{celano:hal-01367280},
  \cite{celano:hal-01382239}),
\end{itemize}

Nous nous sommes aussi intéressés à évaluer les performances de
méthodes de surveillance statistique de procédés dans le cas où:
\begin{itemize}
\item les paramètres nominaux (sous-contrôle) ont été \emph{estimés}
  et ne sont donc pas exactement connus (\cite{chong:hal-01978769},
  \cite{tang:hal-02015138}, \cite{tang:hal-02145998},
  \cite{hu:hal-02155302}, \cite{hu:hal-02160997},
  \cite{castagliola:hal-02190734}, \cite{hu:hal-02318364},
  \cite{khoo:hal-02354567}, \cite{hu:hal-01835841},
  \cite{oprime:hal-01421764}, \cite{you:hal-01614074},
  \cite{teoh:hal-01583912}, \cite{hu:hal-01668745},
  \cite{wu:hal-01286478}, \cite{teoh:hal-01326056},
  \cite{castagliola:hal-01349529}, \cite{yeong:hal-01180304},
  \cite{teoh:hal-01185340}, \cite{you:hal-01216234},
  \cite{you:hal-01348052}) avec les travaux des doctorants invités
  X. Hu et A.A. Tang,
\item une \emph{erreur de mesure} non négligeable doit être prise en
  compte (\cite{sabahno:hal-01977765}, \cite{tang:hal-01978754},
  \cite{sabahno:hal-02190729}, \cite{tang:hal-01806538},
  \cite{sabahno:hal-01921390}, \cite{maleki:hal-01423386},
  \cite{tran:hal-01668732}, \cite{hu:hal-01307056},
  \cite{hu:hal-01327313}, \cite{hu:hal-01396018},
  \cite{hu:hal-01228430}), avec la contribution des doctorants invités
  A. Maleki et H. Sabahno,
\item le calcul des propriétés associés au Run Length doivent être
  calculé de manière exacte, sans approximation ou méthodes de
  simulation (\cite{castagliola:hal-02002980},
  \cite{maravelakis:hal-02022526}, \cite{tang:hal-02059892},
  \cite{khoo:hal-01354062}). 
  \textcolor{red}{FL: je propose de passer ceci en fait marquant:}
  Il faut noter que
  \cite{castagliola:hal-02002980} est le second article publié dans la
  revue Journal of Quality Technology par l'auteur principal et qu'il
  est le seul auteur français à avoir réussi à publier deux articles
  dans cette revue très sélective.
\end{itemize}

Finalement, nous avons contribué à résoudre des problématiques de
dégradation et de maintenance en grandes dimensions appliquées, en
particulier, à des infrastructures du génie civil et utilisant, entre
autre, des méthodes de type Réseaux Bayésiens (RB). Nous nous sommes
plus particulièrement intéressés à:
\begin{itemize}
\item la modélisation pure du phénomène de détérioration et de
  l'aspect cyclique de la phase de maintenance
  (\cite{kosgodagan:hal-01517154}) et la proposition d'un modèle de RB
  dynamique afin de fournir des stratégies de coûts optimisés dans le
  contexte de la gestion de réseaux de ponts routiers
  (\cite{kosgodagandallatorre:hal-01517168}).
\item une extension du RB dynamique standard en introduisant une
  dimension supplémentaire qui prend en compte des ``éléments''
  spatiaux qui sont ainsi liés par un ensemble de covariables qui
  traduisent les dépendances probabilistes existantes
  (\cite{kosgodagan:hal-01517174}). Ce point et le précédent
  correspondent aux travaux de thèse d'A. Kosgodagan.
\item une approche basée sur les RB discrets pour quantifier les
  probabilités de transition de l'état du système
  (\cite{acharige:hal-01152564}),
\item la proposition d'un premier modèle qui considère l'option ou la
  possibilité d'acquérir une information sur la rentabilité de
  nouvelle technologie pas encore disponible sur le marché pour la
  prise de décision de maintenance et remplacement des biens
  (\cite{nguyen:hal-01520843}) avec la thèse de T.P.K. Nguyen,
\item une méthodologie pour la formulation et la construction d'un
  méta-modèle qui intègre des données issues des mesures de contrôle
  non destructif et de l'évolution de la pathologie de dégradation
  choisie (\cite{elhajj:hal-01520822}, \cite{elhajj:hal-01316236})
  avec la thèse de B. El Hajj.
\end{itemize}

 \subsubsection{Conception, planification et ordonnancement des systèmes de production et de services}
  
La recherche menée dans cet axe de l’équipe SLP se structure autour de cinq thématiques majeures : l’ordonnancement, la planification de personnel, la planification de production, la conception de ligne d’assemblage et l’équilibrage de la charge des lignes de production. \\

Concernant l’ordonnancement, les travaux menés se sont focalisés sur la détermination de la complexité de problèmes d’ordonnancement théorique. Ces travaux ont été menés lors de la thèse de doctorat de Tianyu Wang grâce à un financement « Chinese Scholar Council » \cite{wang:hal-02062182,wang:hal-02062178}.  La recherche en planification de personnel a pris comme cadre d’application les systèmes hospitaliers : un projet cofinancé par Atlanstic2020, l’Université d’Angers et IMT Atlantique a permis le démarrage d’une thèse de Doctorat sur le sujet. Il est à noter le dépôt d’un ouvrage aux presses des Mines autour de la planification de personnel. Des travaux autour de la planification des blocs opératoires ont également été menés lors de la thèse de doctorat de Hasan Al Hasan en collaboration avec Université d’Angers et l’Université Catholique de l’Ouest \cite{alhasan:hal-01875359}. \\

Pour la planification de production, les travaux se sont majoritairement concentrés sur deux directions : la planification sous incertitude et la planification financière. La première consiste à définir les quantités à produire ou à commander afin de pouvoir satisfaire la demande en produit fini pour minimiser les coûts logistiques alors que des paramètres tels que la demande, la capacité et le lead time sont incertains. Des modèles mathématiques stochastiques et des méthodes de résolutions efficaces ont été développées pour y répondre \cite{borodin:hal-01313213,benammar:hal-01961194,benammar:hal-01769391,benammar:hal-02435962,benammar:hal-02415341,thevenin:hal-02485676}. La thématique de la planification financière a été abordée lors du projet FUI RCSM et trouve sa continuation dans le cadre du projet ANR FILEAS FOG mené conjointement avec l’IGR-IAE de Rennes. Le but de cette recherche est d'étudier l’impact des décisions financières sur les décisions de planification de la production, et, plus généralement, les décisions logistiques.  Cette recherche a donné lieu à une première thèse de doctorat soutenue par Yuan Bian \cite{bian:halshs-01683781,bian:hal-02190123} et une seconde a débuté en 2018.  Des travaux sur l’intégration de décisions sont également menés : on peut citer par exemple l’intégration de la planification de la production avec les décisions d’inspection \cite{bettayeb:hal-01689377} et la problématique d’intégration des décisions de planification et d’ordonnancement qui ont donné lieu au démarrage d’une thèse CIFRE conjointement avec la société VIF. Outre cela, on peut citer également le développement d’outils de conception/planification et ordonnancement sous incertitudes basées sur des méthodes de programmation stochastique ou robuste, ou encore la résolution du problème de planification intégrée de stock de produit vrac, de routage et d’ordonnancement d’accostage des navires dans un port de vrac dans le cadre d'une collaboration avec l’Université UM6P de Benguerir au Maroc \cite{Najid2020}. \\

La recherche sur la conception combinatoire des systèmes de production reconfigurables est un autre sujet de recherche développé dans l’axe. En se basant sur des résultats obtenus précédemment pour la conception optimale des lignes de transfert pour la fabrication de grandes séries, des modèles et techniques d'optimisation permettant de choisir des équipements et de configurer optimalement des lignes de production reconfigurables pouvant fabriquer plusieurs types de produits ont été développés \cite{battaia:hal-01435089,battaia:hal-01523726,Dolgui2019,Dolgui2020,Battaia2020,yelleschaouche:hal-02485673}. 
\textcolor{red}{FL: phrase longue}\\

Enfin, la dernière thématique concerne l’équilibrage des lignes de production. Celle-ci a été étudiée dans un cas déterministe, en tenant compte des variations de charge induites par les changements de demande, d’un portefeuille multi-produit etc. Des études portant sur la complexité algorithmique de variantes de cette problématique ont été menées et des algorithmes efficaces ont été développés \cite{dolgui:hal-01688688,delorme:emse-01840007}. Cette problématique a également été abordée sous l’angle de la robustesse en utilisant un indicateur appelé « rayon de stabilité » que l’on cherche à maximiser. Les résultats obtenus portent essentiellement sur le développement de méthodes de résolution exactes efficaces à l’aide de la programmation linéaire en nombres mixtes (PLNM) \cite{rossi:hal-01301625}, basée sur la génération de coupes optimums, la réduction de l’espace de recherche grâce à des inégalités valides \cite{pirogov:hal-01614455,pirogov:hal-01832920,pirogov:hal-01713722} ou bien sur la décomposition de Dantzig-Wolf pour de nouvelles variantes du problème de bin-packing qui en découlent \cite{schepler:hal-01518356,schepler:hal-01474542,schepler:hal-02485676}. Une thèse de doctorat a été également soutenu en 2019 sur ce sujet \cite{pirogov:tel-02418792}.
  
\textcolor{red}{FL: est-ce qu'il ne manque pas un certain volume de travaux d'Alexandre ici ?}

  \subsubsection{Conception et optimisation des réseaux logistiques et de transport}
  		
   
Le troisième axe de l'équipe SLP porte sur le développement de modèles et algorithmes d'optimisation pour la conception de réseaux logistiques et la résolution de problèmes de tournées de véhicules (VRP: Vehicle Routing Problem).

Sur le plan scientifique, l'équipe a obtenu des résultats sur les problématiques suivantes : 
\begin{itemize}
	\item Sur les \textit{tournées de véhicules avec synchronisation}, l'équipe a réalisé de fortes avancées sur le développement de la méta-heuristique Large Neighborhood Search (LNS), de procédures de réalisabilité efficaces et sur l'hybridation de LNS avec des approches exactes. Ainsi, en collaboration avec le CIRRELT de Montréal, l'équipe a proposé le premier algorithme intégrant des fenêtre de temps dans le problème de tournées de véhicules (VRP) à deux échelons \cite{grangier2015}. La notion de capacité des points de transfert a été introduite dans le VRP avec cross-docking \cite{grangier:hal-01499170,grangier:hal-02277261} et dans la planification de transports en camions pleins pour les cas de capacité un \cite{grimault:hal-01543503,grimault:tel-01343199}. Ces algorithmes ont été appliqués au transport mixte et multi-modal \cite{masson:hal-01068305}. \textcolor{olive}{au transport mixte passagers-marchandises et au transport multi-modal}
	Ces travaux ont contribué à la soutenance d'une HDR \cite{lehuede:tel-01441778} et ont permis l'obtention de l'ANR PRCI OPUSS. Les premières contributions ont porté sur l'intégration de points et créneaux de livraisons multiples, consignes et préférences  \cite{dumez:hal-02452252}.
	\item Pour la \textit{conception de réseaux logistiques}, l'équipe a réalisé des avancées sur deux principaux aspects: Tout d'abord, l'intégration de critères de développement durables dans la \textit{conception de la supply-chain}, avec un état de l'art orienté optimisation \cite{eskandarpour:hal-01154605} ainsi que la proposition de méta-heuristiques LNS pour la résolution de problèmes mono et multi-objectifs \cite{eskandarpour:hal-01433630, eskandarpour:hal-02407741}.
	L'axe apporte son expertise en conception de la supply-chain dans l'ANR FILEAS FOG à travers l'encadrement de la thèse d'Hamidreza Rezaei, qui s'intéresse à la prise en compte de critères financiers.

    En parallèle, l'équipe développe son expertise sur la \textit{conception de réseaux de transports mutualisés}. Ces travaux ont été initiés pour le secteur de l'horticulture \cite{tang:hal-01591618} avec le FUI Vegesupply. Ils ont été étendus à l'approvisionnement des plate-formes de la grande distribution, à travers la thèse de Juliette Medina \cite{medina:tel-01460708} et en collaboration avec l'université Loyola à Chicago \cite{medina:hal-01689718}. Ils sont poursuivis dans le PIA ADEME OPEN Network sur l'Internet Physique. Ces travaux sont réalisés en partenariat avec l'entreprise 4S-Network et sa filiale CRC Services. Ils se prolongent avec la thèse CIFRE de Gauthier Soleilhac qui intègre le point de vue chargeur. 
    % HLRP 
    L'équipe aborde également les problématiques intégrées de localisation de hubs et routage sous l'angle de leur application aux services postaux \cite{bostel:hal-01320357,yang:hal-02524875} et de la minimisation de l'impact environnemental \cite{yang:these}.
    
    \item Plus généralement, l'équipe contribue à la résolution de problématiques de transport pour l'industrie et les services. Pour le transport de personnes à mobilité réduite, de nouveaux algorithmes ont été proposés dans le CPER NOMAd pour l'optimisation de tournées avec véhicules reconfigurables \cite{tellez:hal-01768291} et la construction de plans de transports réguliers \cite{tellez:hal-02460670}. 
    Ces travaux ont été réalisés dans le cadre du co-encadrement de la thèse d'Oscar Tellez avec l'INSA de Lyon. 
    La collecte de déchets a été abordée dans la thèse CIFRE de Quentin Tonneau \cite{tonneau:hal-01621297, tonneau:tel-01729672} en collaboration avec l'entreprise Brangeon. 
    L'intégration de priorités dans les problèmes de  tournées de véhicules est étudiée dans le cadre de la thèse de Tan Doan en collaboration avec l'université de Hannoï \cite{}. 
    La conception de réseau de bus est étudiée dans les thèses de Ka Yu Lee \cite{lee:hal-01626949} et Hector Gatt en collaboration avec l'entreprise Lumiplan.
    Les problèmes d'équilibrage de la charge des chauffeurs sont étudiés en collaboration avec l'université de Linz \cite{lehuede:hal-02296076}.
\end{itemize}

Dans son activité en transport, l'équipe SLP maintient un équilibre entre  un bon niveau de publications (\textbf{XX} articles en revue) et un nombre important de projets (FUI Végésupply, Thèse 4S-Network, CIFRE Brangeon, deux thèses CIFRE Lumiplan, PHC PROCOPE ATITUDS, PIA ADEME CRC-ON, CPER NOMAD, ANR FILEAS-FOG (inter-axe), ANR OPUSS, CARENE OPT-EMR, Post-Doc UBL, Projet DISC (inter-axe), Atlanstic2020 COLOUR, CIFRE CRC-Services, TSL Cross-regional grant, Mission exploration Japon 2019). 
L'axe a de bonnes collaborations internationales (Montréal, Chicago, Shanghai, Mayence, Linz, Tokyo) et nationales (Lyon, Rennes, Angers). 

Nous sommes de plus visible à travers l'organisation d'évènements (NOW 2015, VEROLOG 2016, ROADEF 2018, Synchrotrans 2019) et notre participation aux groupes de travail du domaine (Board EWGLA 2015-2019, prix de thèse GT2L 2019, prix de thèse VEROLOG 2020).

\subsubsection{Travaux fondamentaux}

L'axe 4 regroupe les travaux de l'équipe avec une coloration plutôt académique portant essentiellement, d'une part, sur l'étude de complexité algorithmique pour des problèmes classiques (ou leurs variantes) en optimisation combinatoire et, d'autre part, sur le développement de méthodes exactes en optimisation multi-objectif.      

\begin{enumerate}
    \item L’équipe a conduit une étude de complexité algorithmique pour des problèmes d'optimisation combinatoire avec données contrôlables et fonctions de risque linéaires (min-$\sum$) et non-linéaires (min-max). Ses problèmes d’optimisation retrouvent des applications en conception du réseau et routage. Pour tous les problèmes traités, nous avons développé des algorithmes polynomiaux et pseudo-polynomiaux plus efficaces \cite{gurevsky:hal-01262639,gurevsky:hal-01272518,gurevsky:hal-01341259,gurevsky:hal-02504487}, en termes de l'ordre de grandeur, par rapport à ceux proposés auparavant dans la littérature. 
    
    \item L’activité sur le thème « optimisation multi-objectif » sur la période 2015-2019 a été naturellement rythmée par le projet ANR-DFG vOpt (Exact Efficient Solution of Mixed Integer Programming Problems with Multiple Objective Functions).


La principale méthodologie considérée dans ce projet pour la résolution exacte de problèmes d’optimisation multi-objectif a été le branch and bound multi-objectif. Un état de l’art détaillé au sujet de ce type de méthode \cite{przybylski:hal-01717951} et ensuite, des travaux raffinant chacun de ses composants ont été réalisés : ensembles bornants, choix de la variable de branchement, choix du noeud actif. 

%Une étude du lien entre l’enveloppe convexe d’un ensemble de points et la décomposition de l’ensemble des poids a permis d’identifier des conditions pour lesquelles des facettes définies par des points non-dominés supportés sont dominées. Ensuite, une méthode s’appuyant sur le calcul incrémental d’un ensemble de points non-dominés supportés, et se reposant fortement sur une initialisation appropriée dans le but d’éviter la considération de facettes dominées, a été proposée [2]. Une nouvelle méthode pour la résolution de la relaxation convexe passant par une décomposition de l’ensemble des poids a également été proposée [3].

Une première méthode pour calculer l’ensemble bornant optimal convexe surrogate a été proposée dans \cite{cerqueus:hal-011583} et repose sur une décomposition de l’ensemble des multiplicateurs. Une nouvelle méthode \cite{przybylski:hal-02480174} a été proposée en réalisant une méthode d’analyse de sensibilité pour le problème dual-surrogate mono-objectif. Cet ensemble bornant supérieur est un outil puissant pour pré-traiter le problème du sac à dos bi-objectif bi-dimensionnel.

Le choix de la variable de branchement a été étudié pour le cas spécifique du sac à dos bi-objectif en variables binaires \cite{cerqueus:hal-01564982}. Le choix du noeud actif a été considéré avec une stratégie originale pour le problème de localisation de services bi-objectif \cite{delmee:hal-01435524} (avec et sans contrainte de capacité). Une autre méthode a  été également proposée pour la résolution exacte du problème avec contrainte de capacité : une méthode en deux phases pour laquelle la seconde phase se repose sur un algorithme de branch and bound \cite{delmee:hal-02480176}. 

\end{enumerate}

  		
\subsection{Faits marquants}



\begin{itemize}
    \item Recrutements d'Alexandre Dolgui (Pr., 2015), Guillaume  Massonnet (MA, 2016), Simon  Thévenin (MA, 2018). Promotion  d'Olivier Péton (Pr., 2017) et de Fabien Lehuédé (Pr., 2019).
    \item Alexandre Dolgui reçoit l'IISE Fellow Award. L'IISE (Institute of Industrial and Systems Engineers) est le principale organisme scientifique en Génie industriel au monde. C'est la première fois qu'un Français reçoit ce prix.
    \item Philippe Castagliola a obtenu une seconde publication dans la revue Journal of Quality Technology \cite{castagliola:hal-02002980}. Il est le seul français à avoir deux publications dans cette revue américaine de très haut niveau.
    \item SLP a obtenu XX publications  dans des revues internationales de rang A.
    \item Organisation de la conférence VeRoLog 2016 (annual workshop of the EURO working group on Vehicle Routing and Logistics optimization). Cette conférence a réuni 200 chercheurs de 20 pays différents sur le site Nantais d'IMT Atlantique. 
    \item Obtention du projet ANR Fileas-Fog (Intégration de la dimension Financière dans l’Optimisation des chaînes loGistiques) en collaboration avec le CREM.
    \item Obtention de l'ANR OPUSS (Optimization of Urban Synchromodal Systems) en collaboration avec l'Université de Mayence (Allemagne).
    \item Alexandre Dolgui est président du comité scientifique internationale de la conférence IFAC MIM 2019 à Berlin (740 participants). 
%    , la conférence é également été sponsorisée par IFIP, IFORS, IISE, INFORMS, ... 10 numéros spéciaux de revues ont été lancés après la conférences avec des version étendues des papiers présentées, les actes sont publiés par Elsevier et sont dans Science direct, Scopus et Web of Science
\end{itemize}


\section{Vie de l'équipe}\label{slp:vieequipe}

SLP se réunit une fois par mois pour un séminaire scientifique où présentent des chercheurs invités (visiteurs, collègues d'équipes du LS2N ou de laboratoires de la région) ainsi que des membres de l'équipe (en priorité les nouveaux arrivants). 
Ces séminaires sont suivi d'une réunion entre permanents. Celle-ci a pour objet la communication et la prise de décisions d'ordre organisationnel. 
A l'été, nous organisons le séminaire annuel de l'équipe sur une ou deux journées.
Celui-ci  privilégie en premier lieu la présentation des nouveaux doctorants et nouveaux projets. C'est également un temps dédié à la réflexion sur le fonctionnement de l'équipe et ses outils.
L'équipe SLP organise également un déjeuner de fin d'année. Les derniers ont notamment permis de réunir l'ensemble de l'équipe et du personnel administratif qui l'entoure.

En 2018 a été organisé un groupe de lecture autour de l'optimisation stochastique. Cinq séances ont eu lieu le midi entre mars et avril. Elle a réuni une dizaine de participants avec pour mission à l'un des participants de présenter un chapitre. Cette initiative demande du temps mais elle a été appréciée et sera reconduite.

La prise collégiale des principales décisions de fonctionnement est privilégiée au sein de l'équipe. A savoir, l'affectation des budgets, les classements de demandes de thèse, les recrutements, font l'objet de discussions entre permanents lors des réunions mensuelles. 
Les projets sont autonomes. Les enseignants-chercheurs déposent leurs projets et disposent de l'intégralité du budget après prélèvements pour la réalisation du projet et de leurs recherches. La mutualisation des reliquats se fait sur la base du volontariat.

Par ailleurs, nous sommes impliqués sous plusieurs aspects dans la vie du laboratoire :
P. Castagliola et A. Dolgui participent au comité de direction du laboratoire. 
O. Bellenguez est responsable adjointe du pôle SDD. 
N. Bostel et A.R. Yelles Chaouche sont membres du conseil de laboratoire. 
E. Gurevsky et O. Péton sont respectivement co-responsables \textbf{(???)}des thèmes transverses Entreprise du futur et Véhicules et mobilités. \textcolor{olive}{ne pas trop insister sur le pole vehicules et mobilites, car pour l'instant rien à signaler} \textcolor{blue}{FL: je n'ai pas l'impression d'insister. Estc-e que co-responsable est le bon terme ?}

L'équipe est également active dans l'animation du RFI Atlanstic 2020. Olivier Péton est membre du collège formation depuis 20XX et du BUREX de ce RFI depuis 2019. Fabien Lehuédé a contribué au montage du partenariat avec IVADO. 

%%%%%%%% ROLE ENSEIGNEMENT ANIMATION
Par ailleurs, l'équipe occupe une place très importante dans l'animation des formations d'IMT Atlantique et de l'Université de Nantes avec notamment: 
\`A IMT Atlantique: A. Dolgui directeur du DAPI, O. Péton, adjoint enseignement du DAPI, O. Bellenguez resp. de l'option GOPL jusqu'en 2019, D. Lemoine resp. de l'option MPR jusqu'en 2020, G. Massonnet resp. de la TAF MPR depuis 2019, T. Yeung resp. du master MOST;
%%%%%
à l'Université: 
X. Gandibleux resp. master Informatique jusqu'en XX ?, 
A. Przybylski resp. parcours ORO, 
%%% A compléter



% Parité ; Intégrité scientifique ; Hygiène et sécurité ; Développement durable et prise en compte des  
% impacts environnementaux ;  Propriété intellectuelle et intelligence économique
