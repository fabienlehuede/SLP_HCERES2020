\section{Projet}
	
	
\def\slp{\textsc{Nouveau Nom}}

	
		\subsection{Analyse SWOT}
		 
		\restylefloat{table}
		 \renewcommand{\labelitemi}{$\bullet$}
		 

			\restylefloat{table}
			\begin{table}[H]
			\begin{tabularx}{\textwidth}{|>{\centering\arraybackslash}l|>{\centering\arraybackslash}X|>{\centering\arraybackslash}X|}
			\hline
			&\textcolor{red}{Positif (pour atteindre l'objectif)} & \textcolor{red}{Négatif (pour atteindre l'objectif)}\\
			\hline
			\color{red}\parbox[t]{2mm}{\rotatebox[origin=rB]{90}{ Origine Interne (organisationnelle) }} & \centering\textcolor{red}{Forces} 
			\smallskip
			\begin{itemize}
			% FORCES 
			\item  Recentrage des thématiques de recherche, ce qui renforce la cohérence de l'équipe.
			\item Les activités de recherche et de valorisation vont souvent de pair : il existe un continuum entre les travaux purement académiques et ceux en lien avec le monde de l'entreprise. 
			
			\end{itemize} ~\\ & \textcolor{red}{Faiblesses}
			\smallskip
			\begin{itemize}
			%FAIBLESSES
			
			\item  texte
			\end{itemize}\\
			\hline
			\centering\color{red}\parbox[t]{2mm}{\rotatebox[origin=rB]{90}{ Origine Interne (organisationnelle) }} & \centering\textcolor{red}{Opportunité} 
			\smallskip
			\begin{itemize} 
			% OPPORTUNITE
			\item  Les sujets de recherche de l'équipe sont au carrefour de plusieurs problématiques majeures, telle que la transition numérique, l'intelligence artificielle, les sciences des données. Notamment, la montée en puissance du numérique dans l'industrie, les services, les transports, rendent accessibles des données qui ne l'étaient pas auparavant. 
			\item  Plusieurs recrutements récents, qui apportent une nouvelle dynamique à l'équipe, et permettent de casser le cloisonnement antérieur de l'équipe en axes trop indépendants.
			\end{itemize} & \textcolor{red}{Menaces}
			\smallskip
			\begin{itemize} 
			\item 
			\item  Départ de Philippe Castagliola et disparition de fait de l'axe Maîtrise des Risques pour les Systèmes Industriels et les Services. L'organisation de l'équipe doit donc être revue. Risque de baisse du niveau de publication. 
			\item Manque de visibilité de la discipline ``recherche opérationnelle" par rapport à la discipline beaucoup plus vaste 
			de l'intelligence artificielle. Ce phénomène n'est pas nouveau. 
			\end{itemize}\\
			\hline
			\end{tabularx}
			\end{table}
		 

		\subsection{Evolution et orientations scientifiques}
		
		
		\subsubsection{Evolution en terme d'effectif et de thématiques}
		
		
		\begin{itemize}
		    \item Départ de Casta, disparition de son axe
		    \item Nouvelle présentation de l'équipe, on abandonne les axes traditionnels
		    \item Organisation selon 3 verrous scientifiques 
		    \item Déclinés en objectifs de contribution
		    \item pour répondre à des défis sociétaux majeurs
		    \item avec des domaines d'application dans les systèmes de production, les chaines logistiques, le transport, les services
		    \item L'activité de chaque membre de l'équipe s'inscrit dans un ou plusieurs défis sociétaux, verrous, objectifs et domaines d'application. 
		\end{itemize}
		
		L'équipe SLP (Systèmes Logistiques et de Production) change donc de nom et devient l'équipe \slp. \textcolor{red}{A compléter}
		
		
		
		\subsubsection{Contexte}
		
		
			\begin{itemize}
		    \item Le chapeau général : transition numérique / industrie du futur
		    \item De nombreuses données sont disponibles, et rendent possibles des projets de recherche et d'innovation qui ne l'étaient pas il y a quelques années
		    \item Cette transition numérique modifie de nombreux modèles d'entreprise, et de nombreux modèles économiques. Essor des pratiques collaboratives, de la mutualisation de moyens.
		    \item Accélération des cycles de vie des produits, et donc nécessité de mettre en place des systèmes de production et des systèmes logistiques flexibles / reconfigurables / adaptables. 
		    \item Prise en compte de l'incertitude, mais aussi de la variabilité des données (pas forcément dans une modélisation stochastique).
		    \item Même dans un univers changeant, certaines décision doivent être prises pour du long terme. Confrontation entre échelle opérationnelle et échelle stratégique. D'où l'intéret de concevoir des systèmes de production ou des systèmes logistiques flexibles, reconfigurables. 
		    \item Prise en compte du risque : aléas = risque faible, mais aussi des risques systémiques = ne pas concevoir de système qui peut s'effondrer
		\end{itemize}
		
		
		
		\subsubsection{Evolution des Thématiques de recherche}
		
	Les thématiques de recherche de l'équipe \slp sont synthétisées par la Figure \ref{fig:projet}. 
	Avec l'abandon de l'Axe Maitrise des Risques pour les Systèmes Industriels, l'équipe n'a plus d'activité dans le domaine de la maitrise statistique des procédés et de la fiabilité. Elle se présente donc aujourd'hui comme une équipe de recherche opérationnelle. 
	
	Nous décrivons par la suite les trois verrous technologiques autour desquels le projet de recherche s'articule. 
	Le premier verrou concerne la résolution de problèmes d'optimisation combinatoire. Ce verrou constitue l'ADN des membres de l'équipe \slp dans la mesure où tous les membres de l'équipe y contribuent. Le deuxième verrou est la résolution de problèmes d'optimisation dans l'incertain, ou avec des données variant dans le temps. Le troisième verrou, appelé modèles enrichis, consiste à modéliser et résoudre des problèmes d'optimisation qui reflètent aussi fidèlement que possible les besoins des utilisateurs finaux et les pratiques émergentes dans les domaines étudiés. Si le premier verrou était déjà présent depuis la création de l'équipe SLP, les deux derniers sont nouveaux et 
	témoignent d'un élargissement des compétences de l'équipe.
	
	
	\subsubsection*{Verrou \# 1 : optimisation combinatoire}
	
	
	
	
	\subsubsection*{Verrou \# 2 : incertitude et variabilité}
	
	\subsubsection*{Verrou \# 3 : modèles enrichis}
	
	
	\subsubsection*{Domaines d'application et défis sociétaux} 
	
	Les domaines d'applications traditionnels de l'équipe étaient les systèmes de production (de biens et de services) et les systèmes logistiques. 
	Ces domaines d'application restent plus que jamais valables et s'inscrivent naturellement dans le cadre de l'Industrie du Futur. 
	Dans les années à venir, nous souhaitons identifier plus clairement notre activité en optimisation des transports, ce qui conduit à la description de quatre domaines d'application principaux : la production de biens, la production de services, les chaines logistiques et les transports. Parallèlement à ces domaines d'application privilégiés, nous avons identifié deux défis sociétaux auxquels nous souhaitons apporter des contributions : la santé du futur et l'efficience énergétique. A noter que ces défis sociétaux peuvent tout à fait se combiner aux domaines d'application. Par exemple, on pourra étudier l'optimisation de réseaux logistiques ou la gestion de la production dans le secteur énergétique, ou bien l'optimisation du transport de patients ou la confection de planning de personnel soignant. 
	
	
	\subsubsection*{Interface avec d'autres disciplines}
	
	\textcolor{bleu}{Remarque : Evacuer tout de suite la question de savoir si la RO fait partie de l'IA, des sciences des données ou de gestion. Quand on parle de la discipline X, il s'agit de tout ce qui n'est pas RO dans cette discipline}
	
	\begin{itemize}
	    \item \textbf{Intelligence artificielle :} les liens sont forts entre la recherche opérationnelle et plusieurs disciplines relevant de l'intelligence artificielle. En particulier, l'équipe utilise déjà des modules relevant de l'apprentissage dans ses algorithmes d'optimisation des transports (\textcolor{blue}{et ailleurs?}. Cette pratique est amenée à se développer dans le futur, notamment dans le développement de méthodes approchées. \textcolor{blue}{quelle utilisation possible du machine learning, d'autres techniques purement IA ? }
	    
	    
	    \item \textbf{Science des données : }
	    
	    Les données deviennent disponibles et abondantes, encore faut-il savoir les traiter. Plusieurs enjeux pour l'équipe \slp : 
	    \begin{itemize}
	        \item définir les données à collecter (en amont d'une étude RO)
	        \item agréger des données brutes pour obtenir des modèles de taille raisonnable (par exemple carroyage en analyse spatiale ou regroupement de clients/ commandes), éliminer les données non pertinentes (en prétraitement)
	        \item 
	    \end{itemize}
	    
	    \item \textbf{Simulation : }
	    
	    En Septembre 2020, recrutement à IMT Atlantique, un poste de maitre de conférences, avec des compétences en IA/données et en simulation. 
	    Objectif = (en plus de l'enseignement) Il ne s'agit pas de faire de la recherche EN simulation, mais de pour mener des recherches couplant recherche opérationnelle et simulation.
	    
	    
	    \item \textbf{Sciences de gestion : }
	    
	    \textcolor{blue}{Dans plusieurs pays Européens, la RO fait partie des "management sciences", et fréqeumment rattachées aux écoles de commerce. Ici, on parlera plutôt des liens potentiels entre notre équipe et des chercheurs en gestion / finance etc.}
	    
	    Continuité des projets RCSM et FILEAS FOG. Les décisions liées à la production et à la logistique ont un impact fort sur les finances de l'entreprise, d'où les travaux communs de ces dernières années, qui ont vocation à être poursuivis. 
	    
	    En France, une partie de la communauté scientifique s'intéressant à la logistique est rattachée à la section CNU 06, et donc en sciences de gestion. Les méthodes scientifiques ne sont pas les mêmes, mais les deux communautés sont complémentaires. Des collaborations ont eu lieu dans le passé (projet OLASI etc.). 
	    
	    Economie : nouveaux modèles économiques. 
	    
	    \item \textbf{Sciences de l'homme : }
	    
	    \textcolor{blue}{A compléter par Odile pour les liens avec sociologie ? }
	    Géographie humaine, et notamment géographie urbaine (aménagement du territoire), géographie de la population, géographie économique (analyse spatiale, accès au ressources. 
	    
	\end{itemize}
	
		
		
\tikzstyle{chapeau} = [rectangle, draw, fill=blue!40, text centered, rounded corners, minimum height=4 em]		
\tikzstyle{chapeau2} = [rectangle, draw, fill=blue!40, text centered, rounded corners]		
\tikzstyle{obj} = [rectangle, draw, fill=blue!20, text centered, rounded corners, minimum height=3 em]
\tikzstyle{detail} = [rectangle, draw, fill=plum!20, text width=3.5cm, text centered, rounded corners, minimum height=4 em]
\tikzstyle{detail1} = [rectangle, draw, fill=red!30,  text centered, rounded corners, minimum height=4 em]
\tikzstyle{detail2} = [rectangle, draw, fill=orange!30, text centered, rounded corners, minimum height=4 em]
\tikzstyle{detail3} = [rectangle, draw, fill=yellow!30, text centered, rounded corners, minimum height=4 em]
\tikzstyle{detail4} = [rectangle, draw, fill=plum!20, text width=2.5cm, text centered, rounded corners, minimum height=4 em]
\tikzstyle{verrou} = [rectangle, draw, text width=9cm, text centered, rounded corners, minimum height=3em]

\tikzstyle{line} = [draw, very thick, color=black!50, -latex']

%{\small
\begin{figure} [htbp]
\label{fig:projet}
\begin{tikzpicture}[scale=0.5, auto]

    %  nodes  
    \node [chapeau, text width=7cm] (RO) at (12,18) {Domaine de recherche: \;\;\;\; recherche opérationnelle}; 
     \node [chapeau,  text width=8.5cm] (IF) at (29,18) {Toile de fond: industrie du futur};

    \node [obj, text width=7cm] (M) at (8,15) {Méthodes};
    \node [obj, text width=3.5cm] (O) at (24,15) {Objectifs de contribution};
        
   %-------------------------------------------------------------
     \node[verrou, fill=red!50] at (15,12) {Optimisation combinatoire};
    \node[detail1, text width=7cm] at (8,9) {Optimisation discrète et mixte};
    \node[detail1, text width=7cm] at (8,6) {Optimisation exacte};
    \node[detail1, text width=7cm] at (8,3) {Méthodes approchées};
    \node[detail1, text width=7cm] at (8,0) {Compléxité};
    
      \node[detail1] at (24,10.5) {Systèmes intégrés, multi-niveaux};
   \node[detail1] at (24,7.5) {Problèmes de grande taille};
   \node[detail1] at (24,4.5) {Synchronisation des systèmes};

    
    %-----------------------------------------------------------
   \node[verrou, fill = orange!50] at (16,-4) {Incertitude et variabilité};     
   \node[detail2, text width=7cm] at (8,-7) {Optimisation robuste};
    \node[detail2, text width=7cm] at (8,-10) {Optimisation stochastique};
    
     \node[detail2] at (24,-7) {Systèmes flexibles, reconfigurables};
    \node[detail2] at (24,-10) {Résilience et robustesse des systèmes};
    \node[detail2] at (24,-13) {Réactivité, systèmes dynamiques};
    \node[detail2] at (24,-16) {Optimisation basée sur les données};

  
  %---------------------------------------------------------
     \node[verrou, fill=yellow!50] at (17,-19) {Modèles Enrichis};

    \node[detail3, text width=7cm] at (8,-22) {Modélisation avancée};
    \node[detail3, text width=7cm] at (8,-25) {Optimisation multiobjectif};
    
    \node[detail3] at (24,-22) {Fonctions objectif alternatives};
    \node[detail3] at (24,-25) {Nouveaux modèles économiques};
    \node[detail3] at (24,-28) {Homme dans la boucle};
    \node[detail3] at (24,-31) {Explicabilité de la décision};
   
   
   %----------------------------------------------- 
   \node [obj, ] (D) at (34,15) {Domaines d'application};
    \node[detail] at (34,12) {Production de biens};
    \node[detail] at (34,8.5) {Production de services};
    \node[detail] at (34,5) {Chaine logistique};
    \node[detail] at (34,1.5) {Transport};

   \node [obj] (D) at (34,-6) {Défis sociétaux};
    \node[detail] at (34,-9) {Santé du futur};
    \node[detail] at (34,-12.5) {Efficience énergétique};

     \node [chapeau2, text width=15.5cm] (I) at (20,-23.5){Interfaces};
    \node[detail4] at (7,-26) {Intelligence artificielle};
    \node[detail4] at (13.5,-26) {Sciences des données};
    \node[detail4] at (20,-26) {Simulation};
    \node[detail4] at (26.5,-26) {Sciences de Gestion};
    \node[detail4] at (33,-26) {Sciences de l'homme};
    



  \end{tikzpicture}
  \caption{Détail du projet d'équipe}
  \end{figure}
  
  
		
		
		
		
		
		
		
		 
